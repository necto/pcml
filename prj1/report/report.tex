\documentclass{article} % For LaTeX2e
% We will use NIPS submission format
\usepackage{nips13submit_e,times}
% for hyperlinks
\usepackage{hyperref}
\usepackage{url}
% For figures
\usepackage{graphicx} 
\usepackage{subfigure} 
% math packages
\usepackage{amsmath}
\usepackage{amsfonts}
\usepackage{amsopn}
\usepackage{ifthen}
\usepackage{natbib}
\usepackage{color}

\title{Project-I by Group Istanbul}

\author{
Johan Droz\\
EPFL \\
\texttt{johan.droz@epfl.ch} \And
Arseniy Zaostrovnykh\\
EPFL \\
\texttt{arseniy.zaostrovnykh@epfl.ch}
}

% The \author macro works with any number of authors. There are two commands
% used to separate the names and addresses of multiple authors: \And and \AND.
%
% Using \And between authors leaves it to \LaTeX{} to determine where to break
% the lines. Using \AND forces a linebreak at that point. So, if \LaTeX{}
% puts 3 of 4 authors names on the first line, and the last on the second
% line, try using \AND instead of \And before the third author name.

\nipsfinalcopy 

\newcommand{\todo}[1]{}
\renewcommand{\todo}[1]{{\color{red} TODO: {#1}}}

\begin{document}

\maketitle

\begin{abstract}
\todo{Discuss and write down the chapter-wise structure of the report.}
\end{abstract}

\section{Data Description}
\todo{probably this one shold remain present.}

\section{Data visualization and cleaning}
\todo{this works as an addition to the previous one.}
% A figure reference example: Figure \ref{fig:boxplotX}

% An example embedded plot:
%\begin{figure}[!t]
%\center
%\subfigure[Boxplot of real-valued $\mathbf{X}$. Data is not centered and therefore we normalize it.]{\includegraphics[width=2.5in]{figures/boxplotX.pdf} \label{fig:boxplotX}}
%\hfill
%\subfigure[Histogram of $\mathbf{y}$. We can clearly see two outliers.]{\includegraphics[width=2.5in]{figures/histY.pdf} \label{fig:histY}}
%\caption{}
%\end{figure}

\section{Regression}
\section{Feature transformations}


\section{Classification}
\section{Feature transformations}


\section{Summary}

\subsubsection*{Acknowledgments}

\subsubsection*{References}

\end{document}
