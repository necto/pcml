\documentclass{article} % For LaTeX2e
% We will use NIPS submission format
\usepackage{nips13submit_e,times}
% for hyperlinks
\usepackage{hyperref}
\usepackage{url}
% For figures
\usepackage{graphicx} 
\usepackage{subfigure} 
% math packages
\usepackage{amsmath}
\usepackage{amsfonts}
\usepackage{amsopn}
\usepackage{ifthen}
\usepackage{natbib}
\usepackage{color}

\title{Project-I by Group Istanbul}

\author{
Johan Droz\\
EPFL \\
\texttt{johan.droz@epfl.ch} \And
Arseniy Zaostrovnykh\\
EPFL \\
\texttt{arseniy.zaostrovnykh@epfl.ch}
}

% The \author macro works with any number of authors. There are two commands
% used to separate the names and addresses of multiple authors: \And and \AND.
%
% Using \And between authors leaves it to \LaTeX{} to determine where to break
% the lines. Using \AND forces a linebreak at that point. So, if \LaTeX{}
% puts 3 of 4 authors names on the first line, and the last on the second
% line, try using \AND instead of \And before the third author name.

\nipsfinalcopy 

\newcommand{\todo}[1]{}
\renewcommand{\todo}[1]{{\color{red} TODO: {#1}}}

\begin{document}

\maketitle

\begin{abstract}
\todo{Discuss and write down the chapter-wise structure of the report.}

\end{abstract}

\section{Data Description}
\todo{probably this one shold remain present.}
\paragraph{Regression} For the regression part of this project we received a dataset containing three named $\boldmath{X\_train, y\_train}$ and $\boldmath{X\_test}$. The pairs $\boldmath{X\_train, y\_train}$  form our training set. 
The former contains N = 2800 sample of dimensionality D = 66 and includes 54 real, 6 binary and 6 categorical variables with a number of categories between 3 and 4.
The later contains 2800 real variables that represent the output of $\boldmath{X\_train}$ .
We also get a test set $X\_test$ of 1200 data examples for which we have to predict the output $\boldmath{y}$ as well as the expected RMSE for our best model.

\section{Data visualization and cleaning}
\todo{this works as an addition to the previous one.}
% A figure reference example: Figure \ref{fig:boxplotX}
\paragraph{Regression set} The dataset for the regression part was analyzed.
We can notice that there are some outliers, so we build a function to remove them and later we will see if this improve the accuracy of our model.
Secondly we noticed that we need to normalized the input data because their distributions are not centered.
Another interesting observation is the distribution of the output $\boldmath{y\_train}$ that is shown on figure \ref{fig:histogramY_train}.

We investigated if dummy coding of the categorical variable can improve the results.

\begin{figure}[h]	
	\centering
	\includegraphics[width=0.9\textwidth]{cluster.pdf}
	\caption{Scatter plot of input vs output}
	\label{fig:scatterIO}
\end{figure}
As we can see on figure \ref{fig:scatterIO}, the data point are grouped into 3 different clusters.


% An example embedded plot:
%\begin{figure}[!t]
%\center
%\subfigure[Boxplot of real-valued $\mathbf{X}$. Data is not centered and therefore we normalize it.]{\includegraphics[width=2.5in]{figures/boxplotX.pdf} \label{fig:boxplotX}}
%\hfill
%\subfigure[Histogram of $\mathbf{y}$. We can clearly see two outliers.]{\includegraphics[width=2.5in]{figures/histY.pdf} \label{fig:histY}}
%\caption{}
%\end{figure}

\section{Regression}
\section{Feature transformations}


\section{Classification}
\section{Feature transformations}


\section{Summary}

\subsubsection*{Acknowledgments}

\subsubsection*{References}

\end{document}
