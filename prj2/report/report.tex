\documentclass{article} % For LaTeX2e
% We will use NIPS submission format
\usepackage{nips13submit_e,times}
% for hyperlinks
%\usepackage{hyperref}
\usepackage{url}
% For figures
\usepackage{graphicx} 
\usepackage{subfigure} 
% math packages
\usepackage{amsmath}
\usepackage{amsfonts}
\usepackage{amsopn}
\usepackage{ifthen}
\usepackage{natbib}
\usepackage{color}
\usepackage{float}
\usepackage{placeins}
\usepackage{geometry}

\title{Project-II by Group Istanbul}

\author{
Johan Droz\\
EPFL \\
\texttt{johan.droz@epfl.ch} \And
Arseniy Zaostrovnykh\\
EPFL \\
\texttt{arseniy.zaostrovnykh@epfl.ch}
}

% The \author macro works with any number of authors. There are two commands
% used to separate the names and addresses of multiple authors: \And and \AND.
%
% Using \And between authors leaves it to \LaTeX{} to determine where to break
% the lines. Using \AND forces a linebreak at that point. So, if \LaTeX{}
% puts 3 of 4 authors names on the first line, and the last on the second
% line, try using \AND instead of \And before the third author name.

\nipsfinalcopy 

\newcommand{\todo}[1]{}
\renewcommand{\todo}[1]{{\color{red} TODO: {#1}}}

\begin{document}

\maketitle

\begin{abstract}
In this report we describe our results for the second project of the 2015 PCML class.
Our goal is to design a system that is able to recognize three kinds of objects present in an image.

\end{abstract}

\section{Data Description}


\section{Data visualization and basic exploratory data analysis}


\subsection{Performance measures}

\section{Predict test data}

%\begin{figure}[!t]
%	\centering
%	\subfigure[Scatter plot of one input variable vs output]{\label{fig:scatter}\includegraphics[width=0.55\textwidth]{figures/X58vsY.pdf}}
%	\subfigure[Histogram of $\boldmath{y\_train}$]{\label{fig:histY}\includegraphics[width=0.4\textwidth]{figures/histY.pdf}}
%	\caption{Data analysis}
%\end{figure}



%\begin{figure}[!t]
%	\center
%	\includegraphics[width=0.8\textwidth]{figures/ridgeRegLoss.pdf}
%	\caption{Plot of the test and training error for ridge regression}
%	\label{fig:ridgeRegError}
%\end{figure}


\section{Summary}

\end{document}
